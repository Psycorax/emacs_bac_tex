\chapter{Einleitung}
Dieses Kapitel gibt einen Überblick über die Inhalte und
Rahmenbedingungen der Arbeit.
\label{cha:Einleitung}
\section{Motivation}
Als Hardware- Software-Entwickler/in wird man im Alltag mit einer
Vielzahl an Programmiersprachen konfrontiert. Ebenfalls reicht ein
einziges Betriebssystem nicht aus, um alle Aufgaben der Arbeitswelt zu
bewältigen. Hinzu kommt noch, dass meist eigene Programme zur
Versionsverwaltung und für die Erstellung von technischen Dokumenten
benötigt werden.

Es gibt einige Möglichkeiten, um diese Vielzahl an Programmiersprachen
auf unterschiedlichen Betriebssystemen zu verwalten. Die erste ist,
für jede Programmiersprache auf jedem Betriebssystem eine eigene
Entwicklungsumgebung zu verwenden. Damit ein effektives Arbeiten
gewährleistet werden kann, muss jede dieser Entwicklungsumgebungen gut
beherrscht werden. Die zweite Möglichkeit ist, einen einfachen Editor
zu wählen, welches auf allen Betriebssystemen installiert ist. Durch
diese einfachen Textverarbeitungsprogramme ist man in vielen Fällen
dazu gezwungen, auf wichtige Features, welche integrierte
Entwicklungsumgebungen mitbringen, zu verzichten. Die dritte
Möglichkeit, welche in dieser Arbeit vorgestellt wird, ist die
Kombination aus den ersten beiden Möglichkeiten. Die Verwendung eines
Editors, welcher auf einer Vielzahl an Betriebssystemen verfügbar
ist. Jedoch muss auf Features wie automatische Vervollständigung,
Syntax-Hervorhebung und Syntax-Überprüfung nicht verzichtet werden.

Bei der Auswahl eines solchen Editors sollte darauf geachtet werden,
dass eine große \textit{Community} dahinter steht. Dadurch müssen die
meisten Anpassungen nicht komplett selbst programmiert werden und es
kann auf Paketen der \textit{Community} aufgebaut werden. Aus diesen
Gründen wird \textit{GNU
  Emacs}\footnote{\url{https://www.gnu.org/software/emacs/}}
gewählt.

Die Konfiguration von Emacs, welche Grundvoraussetzung ist, um Emacs
effektiv in den täglichen Workflow einbinden zu können, erweist sich
meist als sehr aufwändig. Außerdem werden viele Personen durch die
Vielzahl an Tastenkombinationen abgeschreckt. Aus diesem Grund wird
der Einstieg in Emacs oft gemieden, wodurch der Ansporn für diese
Arbeit entstanden ist.\\

\section{Ziel der Arbeit}
Ziel ist es, Informatik-Studierenden einen einfachen Einstieg in Emacs
als Entwicklungsumgebung für mehrere Programmiersprachen zu
ermöglichen. Um dies zu erreichen, werden fertige
Konfigurationsdateien zur Verfügung gestellt und erklärt. Außerdem
wird die Verwendung dieser Konfigurationsdateien beschrieben. Emacs
soll eine alternative zu herkömmlichen Entwicklungsumgebungen bieten,
um ein effektives Arbeiten zu ermöglichen. Damit dies erreicht werden
kann, richten diese Konfigurationsdateien Emacs als vollwertige
Entwicklungsumgebung für diverse Programmiersprachen ein. Diese
Programmiersprachen sind C, C++, Emacs Lisp, Python und VHDL. Die
Konfiguration soll auf mehreren Betriebssystemen lauffähig sein.\\

\section{Gliederung}
Im Kapitel \ref{cha:sw-entwicklung} wird die übliche Vorgehensweise
von Hardware-Software-Entwickler/n/innen dargestellt und welche
Probleme diese mit sich bringen kann. In diesem Kapitel wird ebenfalls
beschrieben wann sich ein Umstieg auf Emacs lohnen kann und wann
nicht. Im Kapitel~\ref{cha:grundlagen} folgt eine Einführung in die
Grundlagen von Emacs. In diesem Kapitel wird auch die Installation von
Emacs auf unterschiedlichen Betriebssystemen behandelt. Im Kapitel
\ref{cha:pakete} werden alle verwendeten Pakete und ihre Funktion
beschrieben. Das Kapitel \ref{cha:Konfiguration} handelt von den
Konfigurationsdateien. Es werden die unterschiedlichen
Konfigurationsdateien erläutert. Im Kapitel \ref{cha:workflow} wird
ein einfacher Workflow für die einzelne Programmiersprachen
beschrieben. Dieses Kapitel soll Emacs-Neulingen ermöglichen, Emacs
direkt in den alltäglichen Workflow einzubauen.\\

\chapter{Die Konfiguration}
\label{cha:Konfiguration}
In diesem Kapitel wird auf die einzelne Konfigurationsdateien
eingegangen. Ebenso werden die Unterschiede der Dateien erklärt.

\section{Die Konfigurationsdateien}
\label{sec:konfigurationsdateien}
Als erstes wird die Datei \textit{init.el} (siehe Abschnitt
\ref{sec:initel}) beschrieben. Diese stellt die Basis dar und wird
direkt von Emacs beim Start als erstes geladen. In dieser Datei kann
eine der folgenden Dateien geladen werden:
\begin{itemize}
\item \textit{myinit\_basic.org} (siehe Abschnitt
  \ref{sec:myinitbasic})
\item \textit{myinit\_coding.org} (siehe Abschnitt
  \ref{sec:myinitcoding})
\end{itemize}

Alle Dateien sind auf \textit{github} unter folgendem Link verfügbar:
\url{https://github.com/Psycorax/emacs_bac}

Die Datei \textit{myinit\_basic.org} enthält alle einfachen
Konfigurationen und Pakete. Diese Einstellungen und Pakete
unterstützen den Benutzer beim Editieren und Programmieren. Es werden
nur {\glqq}leichtgewichtige{\grqq} Pakete geladen. Die
Konfigurationsdatei ist für leistungsschwache Rechner ausgelegt. Sie
funktioniert auf allen gängigen Betriebssystemen, auf denen Emacs
läuft.

Die Datei \textit{myinit\_coding.org} enthält alle Einstellungen der
Datei \textit{myinit\_basic.org}. Zusätzlich enthält sie noch weitere
Konfigurationen und Funktionen. Diese Konfigurationen erweitern
einzelne Modi für die Hardware- und Software-Entwicklung. Die
Konfigurationsdatei enthält Pakete für bessere automatische
Vervollständigung, welche spezifisch an die Programmiersprachen
angepasst sind. Es sind Funktionen zum Erstellen von Ordner-Strukturen
für Hardware- und Software-Projekte enthalten, zum Kompilieren und
Debuggen von Software und zum Kompilieren und Simulieren von
VHDL-Code.\\

\section{init.el}
\label{sec:initel}
Wie bereits erwähnt (siehe Abschnitt \ref{sec:konfdatgl}), startet die
Konfigurationen in der Datei \textit{init.el}. Der komplette Inhalt
der Datei ist auf Seite \pageref{code:initel} zu finden. In der ersten
Zeile steht \texttt{(require 'package)}. Dadurch wird überprüft ob das
Paket {\glqq}package{\grqq} bereits vorhanden ist, ist dies nicht der
Fall so wird es installiert. Das Paket \textit{package} implementiert
ein einfaches System zur Verwaltung von Paketen. Dieses
Verwaltungssystem kann Pakete herunterladen und installieren. Die
Pakete werden automatisch auf den aktuellsten Stand gebracht. In Zeile
zwei steht \texttt{(setq package-enable-at-startup nil)}. Diese Zeile
bedeutet, dass die Variable \textit{package-enable-at-startup} auf den
Wert \textit{nil} gesetzt wird. Durch diese Zeile wird ein
Initialisieren der Pakete nach dem Abarbeiten der Datei
\textit{init.el} beim Start verhindert. Die Pakete werden ohnehin in
Zeile neun manuell initialisiert. In den Zeilen drei bis acht werden
externe Archive für Pakete hinzugefügt.  Folgende externen Archive
werden hinzugefügt:
\begin{itemize}
\item \textit{melpa}
\item \textit{melpa-stable}
\item \textit{org-elpa}
\end{itemize}
Das Archiv \textit{melpa} (siehe Abschnitt \ref{sec:paketebasics})
wurde bereits erwähnt. \textit{Melpa-stable} ist ein Archiv in dem
sich nur von Entwicklern als stabil markierte Pakete befinden. Das
dritte Archiv \textit{org-elpa}, wird für die Installation vom
aktuellen \textit{org-mode} (siehe Abschnitt \ref{sec:orgmode})
benötigt. Unter diesen Zeilen befindet sich, in Zeile neun, das
Kommando \texttt{(package-initialize)}. Dieses leitet die manuelle
Initialisierung der Pakete ein. In den Zeilen 11 bis 13 wird das Paket
\textit{use-package} (siehe Abschnitt \ref{subsec:usepackage}) geladen
und installiert. Unterhalb wird das Paket \textit{use-package} bereits
verwendet. Es wird die aktuellere Version von \textit{org}
installiert. Das Kommando in der letzten Zeile (Zeile 19) lädt nun die
weitere Konfigurationsdatei. In dem folgendem Codestück ist dies die
Datei \textit{myinit\_coding.org}. Das Kommando
\texttt{org-babel-load-file} extrahiert alle Codeblöcke von der
\textit{org}-Datei in eine eigene \textit{el}-Datei und diese wird
dann direkt eingebunden.\\

\begin{program}[ht]
\lstinputlisting[language=Lisp, firstline=1,
  lastline=19]{\string~/.emacs.d/init.el}
\caption{\label{code:initel} Die Datei \textit{init.el}.}
\end{program}

Unter diesen 19 Zeilen Code fügt Emacs beim Start automatisch
generierte Variablen hinzu. Diese sollten nicht verändert
werden. Werden sie gelöscht, dann generiert Emacs die Variablen beim
nächsten Start neu.\\

\section{myinit\_basic.org}
\label{sec:myinitbasic}
Diese Konfigurationsdatei kann auf allen gängigen Betriebssystemen auf
denen Emacs läuft verwendet werden. Es werden keine externen Programme
auf dem System benötigt. In der Datei \textit{myinit\_basic.org}
werden ausschließlich {\glqq}leichtgewichtige{\grqq} Pakete
verwendet. Die Datei ist im Anhang \ref{app:basicorg} abgedruckt,
beginnend auf Seite \pageref{app:basicorg}.\\

\subsection{Verwendete Pakete}
\label{subsec:basicverwpak}
Die Pakete, welche in \textit{myinit\_basic.org} verwendet werden:
\begin{itemize}
\item \textit{avy} (siehe Abschnitt \ref{subsec:avy})
\item \textit{ace-window} (siehe Abschnitt \ref{subsec:acewindow})
\item \textit{try} (siehe Abschnitt \ref{subsec:try})
\item \textit{hungry-delete} (siehe Abschnitt
  \ref{subsec:hungrydelete})
\item \textit{multiple-cursors} (siehe Abschnitt
  \ref{subsec:multiplecursors})
\item \textit{flyspell} (siehe Abschnitt \ref{subsec:flyspell})
\item \textit{undo-tree} (siehe Abschnitt \ref{subsec:undotree})
\item \textit{auto-complete} (siehe Abschnitt \ref{subsec:ac})
\item \textit{org-ac} (siehe Abschnitt \ref{subsubsec:orgac})
\item \textit{magit} (siehe Abschnitt \ref{subsec:magit})
\end{itemize}

\subsection{Zusätzliche Konfigurationen}
\label{subsec:basiczuskonf}
Die Konfigurationsdatei besteht nicht nur aus Paketen und deren
Konfigurationen. Es werden auch noch zusätzliche Konfigurationen an
Emacs selbst vorgenommen werden.\\

\subsubsection{Oberfläche}
Dieser Teil erstreckt sich von Zeile 6 bis 46. In diesem Bereich
werden folgende Konfigurationen vorgenommen:
\begin{itemize}
\item Zeile 10 : Es wird die Startseite von Emacs unterdrückt.
\item Zeile 13 : Verstecken der Symbolleiste.
\item Zeile 16 : Ersetzen der Antwortmöglichkeiten von
  {\glqq}yes{\grqq} und {\glqq}no{\grqq} auf {\glqq}y{\grqq} und
  {\glqq}n{\grqq}.
\item Zeile 24 bis 27 : Abhängig von der verwendeten Emacs-Version
  wird ein Modus verwendet um die Zeilennummern am linken Rand der
  Fenster anzuzeigen. Es ist zu beachten dass der Modus
  {\glqq}global-linum-mode{\grqq} bei langen Dateien sehr viel
  Rechenleistung benötigt. Aus diesem Grund wird er auch ab Emacs 26
  durch den effizienteren Modus
  {\glqq}global-display-line-numbers-mode{\grqq} ersetzt.
\item Zeile 36 : Es wird das vorinstallierte Farbschema
  {\glqq}tango-dark{\grqq} geladen.
\item Zeile 42 : Es wird ein Modus eingeschaltet, welcher die ganze
  Zeile in der sich der Zeiger befindet markiert.
\item Zeile 44 : Die Farbe des Zeigers wird dunkelrot gefärbt.
\end{itemize}

\subsubsection{Globale Tastenkombinationen}
Hier befinden sich einige globale Zuweisungen von Tastenkombinationen
auf Kommandos. In Zeile 51 wird der F5-Taste die Funktion
\textit{revert-buffer} zugewiesen. Diese Funktion lädt den aktuellen
Puffer neu von der Festplatte. In Zeile 52 und 53 wird auf die
Tastenkombination \textbf{\textbackslash e\textbackslash ei} das Öffnen der Datei
\textit{myinit\_basic.org} zugewiesen. Zur Erklärung, \textbf{\textbackslash e}
repräsentiert die Esc-Taste. Also entspricht \textbf{\textbackslash e\textbackslash ei} der
Tastenkombination {\glqq}Esc-Taste Esc-Taste i{\grqq}. Das
Schlüsselwort \textit{interactive} verhindert, dass beim Aufruf der
Funktion ein Argument an das Kommando weitergegeben wird.\\

\subsubsection{Puffer}
Für die Verwaltung von Puffern wird \textit{ibuffer} verwendet (Zeile
59). Es wird die Funktion für die standardmäßige Auflistung von
Puffern mit der Auflistung mittels \textit{ibuffer} überschrieben. Die
Auflistung mittels \textit{ibuffer} kategorisiert die offenen Puffer
und verwendet Farben für die Darstellung.\\

\subsubsection{Fenster}
In Zeile 67 wird der Modus
\textit{winner-mode}\footnote{\url{https://www.emacswiki.org/emacs/WinnerMode}}
aktiviert. Dieser ermöglicht es, alte Fensteranordnungen
wiederherzustellen. Der \textit{winner-mode} ist an die
Tastenkombination \textbf{C-c} gefolgt von der Pfeiltaste nach links
und rechts gebunden.\\

\subsubsection{Navigation}
Für die Vervollständigung im \textit{minibuffer} wird der
\textit{ido-mode}\footnote{\url{https://www.emacswiki.org/emacs/InteractivelyDoThings}}
verwendet. Diese Möglichkeiten zur Vervollständigung werden direkt im
\textit{minibuffer} angezeigt. Der \textit{ido-mode} wird statt der
externen Pakete \textit{ivy} (siehe Abschnitt \ref{subsec:ivy}) und
\textit{counsel} (siehe Abschnitt \ref{subsec:counsel}) verwendet. Die
Zeilen 88 bis 91 aktivieren den \textit{ido-mode}.\\

\subsubsection{Org}
In der Zeile 184 wird die Variable \textit{org-hide-leading-stars} auf
{\glqq}true{\grqq} gesetzt. Dadurch wird nur jeweils der letzte Stern
jeder Überschrift angezeigt, wodurch die Auflistung übersichtlicher
wird.\\

\section{myinit\_coding.org}
\label{sec:myinitcoding}
Die Datei \textit{myinit\_coding.org} enthält erweiterte
Konfigurationen, spezifisch für Hardware- und
Software-Entwickler/innen. Diese Datei ist im Anhang
\ref{app:codingorg} ab Seite \pageref{app:codingorg} zu
finden.\\\\ Für folgende Programmiersprachen werden Konfigurationen
vorgenommen:
\begin{itemize}
\item C
\item C++
\item CMake
\item Python
\item Latex
\item VHDL
\item tcl\\
\end{itemize}
Um alle implementierten Funktionen für C und C++ verwenden zu können,
müssen folgende externe Programme auf dem System installiert sein:
\begin{itemize}
\item \textit{CMake}\footnote{\url{https://cmake.org/}}
\item \textit{gdb}\footnote{\url{https://www.gnu.org/software/gdb/}}\\
\end{itemize}
Um alle implementierten Funktionen für Python verwenden zu können,
müssen folgende externe Programme auf dem System installiert sein:
\begin{itemize}
\item \textit{pyinstaller}\footnote{\url{https://www.pyinstaller.org/}}
\item \textit{elpy} (siehe Abschnitt \ref{subsec:elpy})\\
\end{itemize}
Um alle implementierten Funktionen für Latex verwenden zu können,
müssen folgende externe Programme auf dem System installiert sein:
\begin{itemize}
\item \textit{pdflatex}\footnote{\url{https://www.tug.org/texlive/}}\\
\end{itemize}
Um alle implementierten Funktionen für VHDL verwenden zu können,
müssen folgende externe Programme auf dem System installiert sein:
\begin{itemize}
\item
  \textit{Modelsim}\footnote{\url{https://www.intel.com/content/www/us/en/software/programmable/quartus-prime/model-sim.html}}
\item \textit{ghdl}\footnote{\url{https://github.com/ghdl/ghdl}}\\
\end{itemize}

\subsection{Verwendete Pakete}
\label{subsec:codingverwpak}
In dieser Konfigurationsdatei werden alle in dem Kapitel
\ref{cha:pakete} beschriebenen Pakete verwendet.\\

\subsection{Ähnliche Konfigurationen zur Datei init\_basic.org}
Die folgenden Konfigurationen sind denen der Datei
\textit{init\_basic.org} sehr ähnlich. Es werden nur die Unterschiede
beschrieben:
\begin{itemize}
\item Oberfläche : Das Farbschema {\glqq}tango-dark{\grqq} wird in
  dieser Konfiguration nicht verwendet und ist aus diesem Grund
  auskommentiert. Das selbe gilt für die Farbe des Zeigers.
\item Globale Tastenkombinationen : Es wird zusätzlich an die
  Tastenkombination \textbf{C-x o} das Kommando \textit{ace-window}
  global gebunden. Des weiteren werden noch zusätzlich Tasten mit der
  Funktion \textit{find-file} belegt. Dadurch wird ein schnellerer
  Zugriff auf häufig verwendete Dateien ermöglicht.
\item Puffer : Dieser Teil ist identisch.
\item Fenster : Dieser Teil ist ebenfalls identisch.
\item Navigation : Der Modus \textit{ido-mode} wird nicht verwendet,
  da die Pakete \textit{ivy} und \textit{counsel} diesen Modus
  ersetzten.
\end{itemize}

\subsection{Konfigurationen für grafische Elemente}
\label{subsec:guikonf}
In den Zeilen 900 bis 905 der Konfigurationsdatei können grafische
Elemente ein- und ausgeschaltet werden. Es sollte nur ein Farbschema
zur selben Zeit aktiv sein. Eingeschaltet werden die Elemente, wenn
sie auf {\glqq}t{\grqq} gesetzt werden. Ausgeschaltet werden die
Elemente, wenn sie auf {\glqq}nil{\grqq} gesetzt werden.\\

\subsection{Funktionen für die Softwareentwicklung}
\label{subsec:funkswkonf}
In der Konfigurationsdatei \textit{myinit\_coding.org} sind in den
Zeilen 304 bis 320 zwei Funktionen implementiert. Die Funktion
\textit{this-file-in-dir} vergleicht das Verzeichnis der aktuellen
Datei mit dem übergebenen Parameter. Die Funktion
\textit{default-in-dir} vergleicht die gesetzte Variable
\textit{default-directory} mit dem übergebenen Parameter.

In den Zeilen 365 bis 385 ist die Funktion \textit{beautify-me}
definiert. Diese Funktion richtet den Puffer-Inhalt neu aus, löscht
überflüssige Leerzeichen und ersetzt Tabulatoren durch Leerzeichen
oder umgekehrt. Dies ist von dem mitgegebenen Parameter abhängig. Wird
der Parameter \textit{noTabs} auf {\glqq}t{\grqq} gesetzt, so wird die
Funktion \textit{untabify} ausgeführt, andernfalls
\textit{tabify}. Für Emacs Lisp-Code muss die Funktion
\textit{untabify} verwendet werden, sonnst wird die Ausrichtung der
\textit{hydras} verändert. Aus diesem Grund wurde die Funktion
\textit{beautify-el} explizit definiert.\\

\subsection{Der org- und elisp-Modus}
\label{subsec:orgelispmoduskonf}
Es folgen die expliziten Konfigurationen für den \textit{org-mode} und
\textit{emacs-lisp-mode}.

\subsubsection{hydra}
Für den \textit{org-mode} wird eine \textit{hydra} definiert, diese
hat zu diesem Zeitpunkt nur einen Eintrag. Der Eintrag in der
\textit{hydra} verbindet den Buchstaben {\glqq}b{\grqq} mit einer
zusammengesetzten Funktion. Es wird erst die Funktion
\textit{org-edit-special} ausgeführt, danach die Funktion
\textit{beautify-me} mit dem Parameter auf {\glqq}t{\grqq} gesetzt und
als letztes die Funktion \textit{org-edit-src-exit}. Die Funktion
richtet Codeblöcke aus, ohne diese spezifisch öffnen zu müssen.

Für den \textit{emacs-lisp-mode} werden nicht viele Funktionen
benötigt. Hierdurch wird keine \textit{hydra} definiert. An die
Tastenkombination \textbf{C-c C-b} wird die Funktion
\textit{beautify-el} (siehe Abschnitt \ref{subsec:funkswkonf})
gebunden.

\subsubsection{hooks}
In den \textit{org-mode} sind folgende Modi angehängt:
\begin{itemize}
\item \textit{flyspell-mode} (siehe Abschnitt \ref{subsec:flyspell})
\item \textit{smartparens-mode} (siehe Abschnitt
  \ref{subsec:smartparens})\\
\end{itemize}
In den \textit{emacs-lisp-mode} sind folgende Modi angehängt:
\begin{itemize}
\item \textit{smartparens-mode} (siehe Abschnitt
  \ref{subsec:smartparens})
\item \textit{company-mode} (siehe Abschnitt \ref{subsec:company})
\item \textit{yas-minor-mode} (siehe Abschnitt
  \ref{subsec:yasnippet})\\
\end{itemize}

\subsection{Der C- und C++-Modus}
\label{subsec:cc++moduskonf}
Die beiden Modi \textit{c-mode} und \textit{c++-mode} sind für die
Softwareentwicklung in den Sprachen C und C++ ausgelegt. Die folgenden
Konfigurationen werden in der Konfigurationsdatei
\textit{myinit\_coding.org} durchgeführt.\\

\subsubsection{hydra}
Es ist eine \textit{hydra} mit sieben Einträgen definiert. Diese
Einträge verbinden Tasten mit Funktionalitäten:
\begin{itemize}
\item \textbf{b} : Formatierung des Puffers mit Verwendung von
  Tabulatoren.
\item \textbf{u} : Formatierung des Puffers ohne Verwendung von
  Tabulatoren.
\item \textbf{t} : Erstellen einer Datei mit \textit{ctags} (siehe
  Abschnitt \ref{subsec:ctags}).
\item \textbf{d} : Kompilieren des Projektes und Erstellen einer
  ausführbaren Datei zum Debuggen.
\item \textbf{r} : Kompilieren des Projektes und Erstellen einer
  ausführbaren Datei.
\item \textbf{S} : Startet das Debuggen mit
  \textit{gdb}\footnote{\url{https://www.gnu.org/software/gdb/}}.
\item \textbf{C} : Löschen aller generierten Dateien.
\end{itemize}

\subsubsection{hooks}
In den \textit{c-mode} und \textit{c++-mode} sind folgende Modi
angehängt:
\begin{itemize}
\item \textit{smartparens-mode} (siehe Abschnitt
  \ref{subsec:smartparens})
\item \textit{company-mode} (siehe Abschnitt \ref{subsec:company})
\item \textit{yas-minor-mode} (siehe Abschnitt
  \ref{subsec:yasnippet})\\
\end{itemize}

\subsubsection{Funktionen}
In den Zeilen 460 bis 534 sind Funktionen definiert, welche in der
\textit{hydra} den Tasten zugewiesen werden.\\

\subsubsection{Style}
Für die Modi \textit{c-mode} und \textit{c++-mode}, wird die Funktion
{\glqq}set-my-style-cpp{\grqq} definiert. Diese implementiert einen
Programmierstil und wird an die beiden Modi angehängt. Es wird der
vorgegebene Stil {\glqq}stroustrup{\grqq} verwendet und die
Tabulatorweite auf zwei Leerzeichen
abgeändert\footnote{\url{https://www.emacswiki.org/emacs/IndentingC\#toc2b}}.\\

\subsubsection{Optionale Wahl}
In der Zeile 552 kann die Variable \textit{use\_irony} auf
{\glqq}t{\grqq} gesetzt werden, um das Paket \textit{irony} (siehe
Abschnitt \ref{subsec:irony}) mit Erweiterungen zu benutzen. Dadurch
wird eine spezifische automatische Vervollständigung für C und C++
aktiviert.\\

\subsection{Der Python-Modus}
\label{subsec:pythonmoduskonf}
Der Modus \textit{python-mode} ist für die Softwareentwicklung in der
Sprache Python ausgelegt. Die folgenden Konfigurationen werden in der
Konfigurationsdatei \textit{myinit\_coding.org} durchgeführt.\\

\subsubsection{hydra}
Es ist eine \textit{hydra} mit neun Einträgen definiert. Diese
Einträge verbinden Tasten mit Funktionalitäten:
\begin{itemize}
\item \textbf{b} : Formatierung des Puffers mit Verwendung von
  Tabulatoren.
\item \textbf{u} : Formatierung des Puffers ohne Verwendung von
  Tabulatoren.
\item \textbf{t} : Erstellen einer Datei mit \textit{ctags} (siehe
  Abschnitt \ref{subsec:ctags}).
\item \textbf{p} : Starten einer Python-Shell.
\item \textbf{e} : Inhalt des Puffers an die Python-Shell senden und
  ausführen.
\item \textbf{s} : Python-Shell anzeigen.
  \textit{gdb}\footnote{\url{https://www.gnu.org/software/gdb/}}.
\item \textbf{E} : Ausführbare Datei mit \textit{pyinstaller}
  erzeugen.
\item \textbf{Q} : Beenden der Python-Shell.
\item \textbf{C} : Löschen aller generierten Dateien.
\end{itemize}

\subsubsection{hooks}
An den \textit{python-mode} sind folgende Modi angehängt:
\begin{itemize}
\item \textit{smartparens-mode} (siehe Abschnitt
  \ref{subsec:smartparens})
\item \textit{company-mode} (siehe Abschnitt \ref{subsec:company})
\item \textit{yas-minor-mode} (siehe Abschnitt
  \ref{subsec:yasnippet})\\
\end{itemize}

\subsubsection{Funktionen}
In den Zeilen 617 bis 670 sind Funktionen definiert, welche in der
\textit{hydra} den Tasten zugewiesen werden.\\

\subsubsection{Optionale Wahl}
In der Zeile 675 kann die Variable \textit{use\_elpy} auf
{\glqq}t{\grqq} gesetzt werden um das Paket \textit{elpy} (siehe
Abschnitt \ref{subsec:elpy}) zu benutzen.

\subsection{Der Latex-Modus}
\label{subsec:latexmoduskonf}
Der Modus \textit{latex-mode} ist für das Erstellen von Dokumente in
der Sprache Latex ausgelegt. Die folgenden Konfigurationen sind in der
Konfigurationsdatei \textit{myinit\_coding.org} festgelegt.\\

\subsubsection{hydra}
Es ist eine \textit{hydra} mit zwei Einträgen definiert. Diese
Einträge verbinden Tasten mit Funktionalitäten:
\begin{itemize}
\item \textbf{b} : Kompilieren des Latex-Projektes.
\item \textbf{C} : Löschen aller generierten Dateien.
\end{itemize}

\subsubsection{hooks}
An den \textit{latex-mode} sind folgende Modi angehängt:
\begin{itemize}
\item \textit{flyspell-mode} (siehe Abschnitt \ref{subsec:flyspell})
\item \textit{smartparens-mode} (siehe Abschnitt
  \ref{subsec:smartparens})
\item \textit{company-mode} (siehe Abschnitt \ref{subsec:company})
\item \textit{yas-minor-mode} (siehe Abschnitt
  \ref{subsec:yasnippet})\\
\end{itemize}

\subsubsection{Funktionen}
In den Zeilen 715 bis 776 sind Funktionen definiert, welche in der
\textit{hydra} verwendet werden.\\

\subsection{Der VHDL-Modus}
\label{subsec:vhdlmoduskonf}
Der Modus \textit{vhdl-mode} ist für die Hardwarebeschreibungssprache
VHDL ausgelegt. Die folgenden Konfigurationen werden in der
Konfigurationsdatei \textit{myinit\_coding.org} durchgeführt.\\

\subsubsection{hydra}
Es ist eine \textit{hydra} mit sechs Einträgen definiert. Diese
Einträge verbinden Tasten mit folgenden Funktionalitäten:
\begin{itemize}
\item \textbf{b} : Formatierung des Puffers mit dem vorhandenem
  \textit{beautifier}\footnote{\url{http://doc.endlessparentheses.com/Fun/vhdl-beautify-buffer.html}}
  im \textit{vhdl-mode}.
\item \textbf{c} : Kompilieren des Projektes mit \textit{ghdl}.
\item \textbf{s} : Ein \textit{tcl}-Datei festlegen, mit welchem
  \textit{Modelsim} gestartet wird.
\item \textbf{S} : Starten von \textit{Modelsim} mit der gesetzten
  \textit{tcl}-Datei.
\item \textbf{t} : Erstellen einer Datei mit \textit{ctags} (siehe
  Abschnitt \ref{subsec:ctags}).
\item \textbf{C} : Löschen aller generierten Dateien.
\end{itemize}

\subsubsection{hooks}
An den \textit{vhdl-mode} und \textit{tcl-mode} sind folgende Modi angehängt:
\begin{itemize}
\item \textit{smartparens-mode} (siehe Abschnitt
  \ref{subsec:smartparens})
\item \textit{company-mode} (siehe Abschnitt \ref{subsec:company})
\item \textit{yas-minor-mode} (siehe Abschnitt
  \ref{subsec:yasnippet})\\
\end{itemize}

\subsubsection{Funktionen}
In den Zeilen 808 bis 885 sind Funktionen definiert, welche in der
\textit{hydra} den Tasten zugewiesen werden.\\

\subsubsection{Optionale Wahl}
In der Zeile 675 kann die Variable \textit{use\_elpy} auf
{\glqq}t{\grqq} gesetzt werden um das Paket \textit{elpy} (siehe
Abschnitt \ref{subsec:elpy}) zu benutzen.

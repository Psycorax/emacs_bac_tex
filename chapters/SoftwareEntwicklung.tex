\chapter{Gegenüberstellung}
\label{cha:sw-entwicklung}
Dieses Kapitel stellt die Ausgangslage als Hardware- und
Software-Entwickler/in dar. Es wird beschrieben, in welchen Fällen
sich ein Umstieg auf Emacs lohnt.

\section{Herkömmliche Entwicklungsumgebungen für unterschiedliche Aufgaben}
In den folgenden Abschnitten werden übliche Entwicklungsumgebungen für
Hardware- und Software-Entwickler/innen aufgelistet. Im Anschluss
werden die möglichen Probleme, die diese Vielzahl an
Entwicklungsumgebungen mit sich bringen kann, zusammengefasst.\\

\subsection{Entwicklungsumgebungen zur Erstellung von Dokumenten}
In der Informatik werden einige Programme benötigt, um ein Projekt
effektiv ausarbeiten zu können. Zunächst wird eine Software für die
Erstellung technischer Dokumente benötigt. Diese Dokumente reichen von
Pflichtenheften bis hin zu Softwarebeschreibungen. Dafür eignet sich
die Sprache LaTeX, ein Editor für LaTeX ist zum Beispiel
\textit{MikTex}\footnote{\url{https://miktex.org/}}. Eine weitere
Optionen ist Software aus dem Paket \textit{Microsoft
  Office}\footnote{\url{https://www.office.com/}}, wie
\textit{Microsoft
  Word}\footnote{\url{https://products.office.com/de-at/word}} oder
\textit{Microsoft
  Excel}\footnote{\url{https://products.office.com/de-at/excel}} für
die Erstellung von Tabellen.\\

\subsection{Entwicklungsumgebung zum Entwickeln von Code}
Als nächstes wird eine Entwicklungsumgebung für die jeweils verwendete
Programmiersprache ausgewählt. Eine kostenlose Variante ist
\textit{Eclipse}\footnote{\url{https://www.eclipse.org/}}. Es kann
aber auch eine Version von \textit{Visual
  Studio}\footnote{\url{https://visualstudio.microsoft.com/}}
verwendet werden. Es muss bei der Wahl der Entwicklungsumgebung auf
folgende Punkte geachtet werden:
\begin{itemize}
\item Auf welchem Betriebssystem wird entwickelt.
\item Für welche Plattform wird entwickelt.
\item Welche Programmiersprache wird verwendet.
\item Wenn mehrere Programmiersprachen zum Einsatz kommen, muss eine
  Entwicklungsumgebung gefunden werden, die all diese Sprachen
  unterstützt. Andernfalls müssen mehrere Umgebungen verwendet werden.
\item Unterstützt die Entwicklungsumgebung den erforderlichen
  Compiler.
\end{itemize}

\subsection{Sonstige zusätzliche Software}
Zur Versionsverwaltung wird in einigen Fällen eine grafische
Oberfläche bevorzugt. Für Versionsverwaltungssysteme wie
\textit{Git}\footnote{\url{https://git-scm.com/}} und
\textit{Subversion}\footnote{\url{https://subversion.apache.org/}}
gibt es zum Beispiel \textit{Tortoise
  Git}\footnote{\url{https://tortoisegit.org/}} und \textit{Tortoise
  SVN}\footnote{\url{https://tortoisesvn.net/}}.\\

\subsection{Probleme bei einer großen Anzahl an Software}
Wird an einem Projekt für einen großen Zeitraum gearbeitet und für
dieses Projekt wurden eine oder zwei Entwicklungsumgebungen
ausgewählt. Dann ist nichts dagegen einzuwenden, diese
Entwicklungsumgebungen zu erlernen und zu verwenden. Jedoch gibt es
ebenfalls Projekte, die nur von kurzer Dauer sind. Es muss jeder
persönlich abschätzen, ob es Sinnvoll ist eine neue
Entwicklungsumgebung für diesen kurzen Zeitraum zu erlernen.

Ebenfalls kann es vorkommen, dass für die Entwicklung einiger
Abschnitte des Projektes ein Server zur Verfügung gestellt wird, auf
welchem eine Distribution von
\textit{Linux}\footnote{\url{https://www.kernel.org/}} läuft. Dies ist
problematisch, wenn die persönliche Wahl der Entwicklungsumgebung auf
eine Windows-basierte Entwicklungsumgebung fiel und diese für Linux
nicht verfügbar ist. So muss wieder eine neue Entwicklungsumgebung für
die erforderliche Aufgabe gesucht und erlernt werden. Andernfalls muss
man sich mit einem einfachen Editor zufrieden geben. Weiter Probleme
können auftreten, wenn Programmierer am selben Projekt auf
unterschiedlichen Betriebssystemen arbeiten. In einigen Fällen hat man
selber auch gar keine Wahl, welche Entwicklungsumgebung zu verwenden
ist. Zum Beispiel wenn man einem Projekt zugewiesen wird, für das eine
Entwicklungsumgebung bereits gewählt wurde. Hier kann wiederum auf
einen einfachen Editor ausgewichen werden oder man lernt, mit der
Entwicklungsumgebung zu arbeiten.\\

\section{Warum Emacs}
Emacs ist einer der mächtigsten Editoren und eine eigenständige
Arbeitsumgebung. Damit ist gemeint, dass Emacs auch Dateien
umbenennen, verschieben, kopieren und löschen kann. Wenn Emacs in
einer grafischen Umgebung gestartet wird, können grafische Objekte wie
Bilder angezeigt werden. Wird Emacs in einem Terminal ausgeführt, so
kann zwar nur Text angezeigt werden, aber Emacs kann so auch auf einem
Server ohne grafischer Oberfläche verwendet werden. Emacs ist ein
Emacs, welches jedoch in einem Programmiergerüst einem so genanntem
{\glqq}Framework{\grqq} läuft. Dieses Programmiergerüst ist höchst
personalisierbar, wodurch integrierte Entwicklungsumgebungen
geschaffen werden können, die an die eigenen Bedürfnisse angepasst
sind. Wird Emacs unter
\textit{Unix}\footnote{\url{http://www.opengroup.org/unix}} verwendet,
so können alle \textit{Unix}-Kommandos innerhalb von Emacs ausgeführt
werden. Das selbe gilt auch für
\textit{Mac-OS}\footnote{\url{https://www.apple.com/at/macos/high-sierra/}}
und \textit{Windows}. Es ist dabei zu beachten, dass die
\textit{Windows}-Kommandozeile bei weitem nicht so mächtig ist, wie
die \textit{Unix}-Kommandozeile. Die
\textit{Powershell}\footnote{\url{https://docs.microsoft.com/en-gb/powershell/}}
ist eine beachtliche Verbesserung zur \textit{Windows}-Kommandozeile,
kann jedoch ebenfalls nicht mit der \textit{Unix}-Kommandozeile
mithalten. Dies ist jedoch auf das Betriebssystem und die Tatsache,
dass die \textit{Unix}-Kommandozeile alle In- und Outputs als Dateien
interpretiert, zurückzuführen. \cite{CameronRosenblattRaymond1996}\\

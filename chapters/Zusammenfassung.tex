\chapter{Zusammenfassung}
\label{cha:Zusammenfassung}
\section{Aufgetretene Probleme}
\label{sec:probleme}
Anfangs wurden für die Arbeit Pakete evaluiert, welche ganze
Entwicklungsumgebungen für C und C++ implementieren. Zum Beispiel das
Paket
\textit{cmake-ide}\footnote{\url{https://github.com/atilaneves/cmake-ide}}. Diese
kompletten Entwicklungsumgebungen funktionieren grundsätzlich sehr
gut. Das Problem an diesen fertigen Entwicklungsumgebungen ist, dass
sie meist eine große Anzahl an externen Programme benötigen. Ein
weiteres Problem ist, dass diese Pakete für die unterschiedlichen
Sprachen auf einer Vielzahl an Paketen basieren. Diese Pakete
implementieren meist jedoch grundlegend die selbe Funktionalität.

Die Schwierigkeit bestand nun darin, Pakete zu finden, welche für alle
verwendeten Sprachen verwendet werden können. Diese Pakete sollten
ausschließlich Software verwenden, welche auf mehreren
Betriebssystemen verfügbar ist.

\section{Umsetzung}
\label{sec:umsetzung}
Das Ziel dieser Arbeit war es, einfache Entwicklungsumgebungen für
Emacs zu finden. Diese Entwicklungsumgebungen sollten für mehrere
Programmiersprachen (C, C++, Python, Latex, CMake, Emacs Lisp und
VHDL) parallel verwendbar sein. Die Entwicklungsumgebungen für die
unterschiedlichen Sprachen sollten ähnlich bedienbar sein und auf den
selben Paketen aufsetzten.

Um diese einfachen und ähnlich bedienbaren Entwicklungsumgebungen
realisieren zu können, wurden eigene Funktionen für die einzelnen Modi
definiert. Diese Funktionen bauen auf den gewählten Paketen auf und
erweitern diese. Es wurde folgende Funktionalität implementiert:
\begin{itemize}
\item Die Erstellung von Projekverzeichnissen,
\item Die Ausrichtung und Formatieren von Code,
\item Das Kompilieren des Projektes,
\item Das Erzeugen von ausführbaren Dateien,
\item Das Aufräumen von Projekten, durch entfernen generierter Ordner
  und Dateien,
\item Das Starten des Debuggens für Projekte und
\item Das Starten von Simulationen in externer Software.\\
\end{itemize}
Damit diese Funktionen für den Benutzer leicht zugänglich sind, werden
sie in \textit{hydras} (siehe Abschnitt \ref{subsec:hydra})
hinterlegt.

\section{Mögliche Erweiterungen}
\label{sec:erweiterungen}
Eine mögliche Erweiterung dieser Arbeit ist das Einbinden von Paketen,
welche spezifisch für die Entwicklung auf eingebetteten Systeme
ausgelegt sind. Ein mögliches Paket an dem angesetzt werden kann ist
\textit{stm32-emacs}\footnote{\url{https://github.com/SL-RU/stm32-emacs}}. Dieses
Paket baut beinahe ausschließlich auf Software auf, welche bereits für
die Pakete in dieser Arbeit benötigt wurden.

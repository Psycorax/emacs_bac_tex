\chapter{Kurzfassung}
\label{cha:kurzfassung}
Als Hardware- und Software-Entwickler/in ist es üblich, eine große
Anzahl an Entwicklungsumgebungen für einzelne Programmiersprachen zu
verwenden. Häufig findet die Beschreibung der Hardware oder die
Implementierung der Software auf unterschiedlichen Betriebssystemen
statt. Dadurch kommen in vielen Fällen noch weitere
Entwicklungsumgebungen hinzu. Damit diese Aufgaben mit einer einzigen
Software bewältigt werden können, wird der Editor Emacs
vorgestellt.\\\\ Diese Arbeit beschreibt, wie Emacs als Alternative
für die Hardware- und Software-Entwicklung eingesetzt werden
kann. Dafür wird Emacs spezifisch an die Anforderungen eines Hardware-
und Software-Entwicklers angepasst. Im Zuge der Arbeit wird
beschrieben, wie diese Anpassungen durchgeführt werden. Die
Konfigurationen setzten Emacs als eigenständige Entwicklungsumgebung
für die Sprachen C, C++, Emacs Lisp, Latex, Python und VHDL auf. Zur
Realisierung werden Pakete verwendet, welche von der Community zur
Verfügung gestellt werden. Die Funktionalität und Konfiguration aller
verwendeten Pakete werden beschrieben. Für diese Pakete werden eigene
Funktionen implementiert, damit die Funktionalität der Pakete für
mehrere Sprachen möglichst einfach genutzt werden kann.\\\\ Es werden
ebenfalls Pakete vorgestellt, welche in Emacs eine grafische
Oberfläche implementieren. Mit diesen grafischen Paketen sieht Emacs
ansprechender aus. Dadurch wird Personen, die grafische Oberflächen
bevorzugen, Emacs für die Hardware- und Software-Entwicklung
schmackhaft gemacht. Damit auch Emacs-Neulingen ein einfacher Einstieg
ermöglicht wird, werden die Grundlagen von Emacs beschrieben.\\\\ In
der Arbeit werden Beispiele von Workflows für die einzelnen
Programmiersprachen beschrieben. Dadurch kann Emacs als
Entwicklungsumgebung schnell und einfach in den Alltag eine/s/r
Hardware- und Software-Entwickler/s/in integriert werden.\\
